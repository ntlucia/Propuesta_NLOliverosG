\documentclass[11pt]{article}

\usepackage{amsmath}
\usepackage{amssymb}
\usepackage[utf8]{inputenc}
%\usepackage[latin1]{inputenc}
\usepackage[spanish]{babel}
\usepackage[left=3cm,right=3cm,top=3cm,bottom=2.5cm]{geometry}
\usepackage{amsmath,amssymb,latexsym,color,graphicx,verbatim}
\usepackage{mathrsfs}
\usepackage{layout}
\usepackage{graphicx}
\usepackage{multirow}
\usepackage[table,xcdraw]{xcolor}
%COLOCA COMANDOS EN ESPAÑOL
%\renewcommand{\contentsname}{Contenido}
%\renewcommand{\partname}{Parte}
%\renewcommand{\appendixname}{Apéndice}
%\renewcommand{\figurename}{Figura}
%\renewcommand{\tablename}{Tabla}
%\renewcommand{\abstractname}{Resumen}
%\renewcommand{\refname}{Bibliografía}
%FIN DEL BLOQUE
\usepackage[acronym,shortcuts]{glossaries} %PARA UN GLOSARIO DE ACRÓNIMOS
\makeglossaries
\usepackage[font=small]{caption}
\usepackage[colorlinks = true,
                     linkcolor = blue,
                     citecolor = red,
                     urlcolor = blue]{hyperref}

\baselineskip0.75cm
\parskip0.5cm
\parindent0cm

\begin{document}


\begin{titlepage}
\centering {\Large {\sc Estudio del último eclipse cromosférico de Zeta Aurigae, otoño 2019}}

\vfill
\centering {\Large Propuesta de trabajo de grado para optar al t\'itulo de F\'isica}
\vfill
\hfill

\centering {\Large Natalia Lucía Oliveros Gómez$^{1,2}$}

\vfill

\centering {\Large Director: Klaus-Peter Schröder $^{3}$}

\centering {\Large  Co-Director: Luis Alberto Núñez$^{1,2}$}

\centering {\Large  Co-Director: Faiber Daniel Rosas$^{3}$}


\hfill




{{\Large $^1$Grupo de Investigaci\'on en Relatividad y Gravitaci\'on GIRG}} \\
{{\Large$^2$Grupo Halley de Astronom\'ia y Ciencias Aeroespaciales}} \\
{{\Large$^3$Universidad de Guanajuato}} \\

\vfill
\vfill

\centering {\Large Universidad Industrial de Santander\\Facultad de
Ciencias\\Escuela de F\'{i}sica\\Bucaramanga\\2018}


\end{titlepage}


\newpage

\tableofcontents

%\newpage

%\acrodef{LAGO}{Latin American Giant Observatory}

\newpage




%%%%%%%%%%%%%%%%%%%%%%%%%%%%%%%%%%%%%%%%%%%%%%%%%%%%%%%%%%%%%%%%%%%%%%%%%%%%%%%%%%%%%%%%%%%%%%

\begin{table}[htbp]
\begin{center}
\resizebox{14cm}{!}{
\begin{tabular}{|l|l|l|l|l|}
\hline
\multicolumn{5}{|l|}{\begin{tabular}[c]{@{}l@{}}\textbf{Título de la propuesta:}\\ Estudio del último eclipse cromosférico de Zeta Aurigae, otoño 2019\end{tabular}}           \\ \hline
\multicolumn{5}{|l|}{\begin{tabular}[c]{@{}l@{}}\textbf{Nombre del estudiante:}\\ Natalia Lucía Oliveros Gómez\end{tabular}}                                                    \\ \hline
\textbf{Código:} 2160778                          & \multicolumn{3}{l|}{\textbf{E-mail:} onatalialucia@gmail.com}                              & \textbf{Cel:} 3123154756                         \\ \hline
\multicolumn{5}{|l|}{\begin{tabular}[c]{@{}l@{}}\textbf{Nombre del grupo de Investigación:}\\ Universidad de Guanajuato\end{tabular}}                                           \\ \hline
\multicolumn{5}{|l|}{\textbf{Dirrección:} Guanajuato, Guanajuato, México}                                                                                                       \\ \hline
\textbf{Tel:}                                     & \multicolumn{4}{l|}{\textbf{E-mail:}}                                                                                                \\ \hline
\multicolumn{5}{|l|}{\textbf{Líneas de invest}igación desarrolladas por el grupo:}                                                                                              \\ \hline
\multicolumn{5}{|l|}{\begin{tabular}[c]{@{}l@{}}\textbf{Profesor de la Escuela de Física que dirigirá el trabajo:}\\ Luis Alberto Núñez de Villavicencio Martínez\end{tabular}} \\ \hline
\multicolumn{5}{|l|}{\begin{tabular}[c]{@{}l@{}}\textbf{Profesional del grupo de investigación que servirá de tutor:}\\ Klaus-Peter Shrörder\end{tabular}}                      \\ \hline
\end{tabular}}
\end{center}
\end{table}

\newpage
%%%%%%%%%%%%%%%%%%%%%%%%%%%%%%%%%%%%%%%%%%%%%%%%%%%%%%%%%%%%%%%%%%%%%%%%%%%%%%%%%%%%%%%%%%%%%%
\begin{abstract}
Este proyecto tiene como objetivo comparar los espectros de absorción de la cromosfera y observar cómo durante el eclipse hay un cambio en la densidad de columna que depende de la altura, con antiguos datos de eclipses del mismo sistema binario y así poder demostrar la dinámica de la cromosfera de las estrellas.

\vspace{0.5cm}
\textbf{Palabras clave:} Estrellas binarias eclipsante, espectro cromosférico, Curvas de crecimiento.

\end{abstract}


\begin{figure}
  \centering

  \label{Figura 1}
\end{figure}


\section{Grupo de Investigación de la Universidad de Guanajuato}

\section{Lugar dentro de los proyectos que tiene el grupo de investigación}

\section{Problema de investigación}
%%%%%%%%%%%%%%%%%%%%%%%%%%%%%%%%%%%%%%%%%%%%%%%%%%%%%%%%%%%%%%%%%%%%%%%%%%%%%%%%%%%%%%%%%%%%%%
\section{Justificación}
Zeta Aurigae es un sistema binario eclipsante de estrellas ubicado en la constelación de Auriga, el cual tiene un plano orbital que coincide con la línea de visión desde la tierra, además este se puede ver a simple vista, así que se observa que durante los eclipses la magnitud disminuye +3,99. Este sistema resulta relevante, por el hecho de ser el primero en ser consignado como un sistema binario espectroscópico. La estrella Zeta Aurigae A es una supergigante roja tipo K4 II y Zeta Aurigae B es una estrella de secuencia principal tipo B7 V.
\vspace{3mm}

En artículos y estudios anteriores se ha considerado una ley de potencias de forma exponencial (ley barométrica exponencial) [CITA KLAUS 10] que relaciona la escalas de densidad de columna y  altura se podía considerar contante (tomando un promedio) ya que aunque tenía un crecimiento, este era muy lento y por lo tanto se podía despreciar, la cual se podía ver como

\begin{equation}
    n(h) = n_o \exp{(-h/\alpha)}
\end{equation}{}
\vspace{2mm}

Sin embargo en espectros actuales, con mejor resolución y mayor relación S/N se observa que estos cambios si afectan posibles conclusiones de la dinámica de la cromosfera de las estrellas, por lo cual podría ser necesario considerar otro módelo de densidad que se ajuste mejor a los parámetros geométricos del eclipse y tenga en cuenta dichas variaciones y las incertidumbres disminuyan.
%%%%%%%%%%%%%%%%%%%%%%%%%%%%%%%%%%%%%%%%%%%%%%%%%%%%%%%%%%%%%%%%%%%%%%%%%%%%%%%%%%%%%%%%%%%%%%
\section{Objetivos}

%%%%%%%%%%%%%%%%%%%%%%%%%%%%%%%%%%%%%%%%%%%%%%%%%%%%%%%%%%%%%%%%%%%%%%%%%%%%%%%%%%%%%%%%%%%%%
\textbf{Objetivo General}

\begin{itemize}
\item Comparar la absorción cromosférica y el cambio de la densidad de columna N(h) del eclipse de otoño 2019 con un antiguo eclipse de 1987.
\end{itemize}
\textbf{Objetivos Espec\'ificos}
\begin{enumerate}
    \item Entender cómo reducir datos de espectros

    \item Construir curvas de crecimiento experimentales

    \item Comparar curvas de crecimiento del eclipse otoño 2019 y eclipse 1987
\end{enumerate}

%%%%%%%%%%%%%%%%%%%%%%%%%%%%%%%%%%%%%%%%%%%%%%%%%%%%%%%%%%%%%%%%%%%%%%%%%%%%%%%%%%%%%%%%%%%%%%

\newpage

%%%%%%%%%%%%%%%%%%%%%%%%%%%%%%%%%%%%%%%%%%%%%%%%%%%%%%%%%%%%%%%%%%%%%%%%%%%%%%%%%%%%%%%%%%%%%%

\section{Metodolog\'ia}

Para alcanzar los objetivos propuestos, se seguirá el siguiente orden de actividades:

\begin{itemize}

\item[1.1] Instalar y aprender a usar softwares para análisis de espectros:
\begin{itemize}
    \item iSpec: Genera síntesis espectrales, permite encontrar parámetros estelares mediante ajuste de líneas espectrales y ajuste del continuo usando modelos.
    \item Código que permite analizar la absorción de la línea K de Ca II, permite medir el ancho de línea y el área bajo la curva
    \item Código de evolución estelar: Genera trayectorias evolutivas para determinar masas y edades de las estrellas en un diagrama HR.
\end{itemize}
\item[1.2] Clasificar la lineas y su cromosfera de la estrella gigante y la secundaria
\item[1.3] Obtener un registro de los espectros en orden temporal y calcular en cada un caso la altitud proyectada de la compañera sobre la fotosfera del gigante

\item[1.4]  Usar el espectro puro de la gigante para durante la totalidad del eclipse poder sustraer los espectros compuestos para obtener y así obtener el espectro puro la compañera, en el cual se vean las lineas cromosféricas

\item[1.5] Medir el ancho equivalente de las líneas cromosféricas (área de línea de absorción dividido por el continuo local)

\item[2.1] Lecturas de artículos en los que hayan resultos procedimientos similares

\item[2.2] Graficar curvas de crecimiento para cada sesión del eclipse (estrella individual, parcial 'tanto entrada como salida', total). Además estas se hacen para el hidrógeno y para el Ca II 
\item[3.1] Comparar diferentes parámetros (\textit{ionización, temperaturas de excitación,ancho doppler,número de átomos, velocidades radiales}) con la densidades de columna N(h)

\item[3.2] Deducir un sencillo modelo de la densidad de columna en cada un caso

 
\end{itemize}

Con estas actividades voy a poder desarrollar ciertas competencias que resultan muy útiles e importantes para trabajos a futuro:

\begin{itemize}
    \item Identificación de líneas espectrales y clasificación de espectros
    \item Uso de gráficas basado en plataformas de linux
    \item Entender y aplicar la física espectroscopia de líneas de absorción
    \item Obtener un fondo de la física de atmósferas estelares
\end{itemize}


\section{Resultados esperados}
%%%%%%%%%%%%%%%%%%%%%%%%%%%%%%%%%%%%%%%%%%%%%%%%%%%%%%%%%%%%%%%%%%%%%%%%%%%%%%%%%%%%%%%%%%%%%%

\section{Condiciones y recursos que ofrece el grupo de investigación}
%%%%%%%%%%%%%%%%%%%%%%%%%%%%%%%%%%%%%%%%%%%%%%%%%%%%%%%%%%%%%%%%%%%%%%%%%%%%%%%%%%%%%%%%%%%%%%
\section{Cronograma de Actividades}	
\begin{table}[htbp]
\resizebox{17cm}{!}{
\begin{tabular}{|l|c|l|l|l|l|l|l|l|l|l|l|l|l|l|l|l|l|l|l|l|l|l|l|l|}
\hline
\multicolumn{1}{|c|}{}                                                              &                                                                              & \multicolumn{3}{c|}{Marzo}                                                     & \multicolumn{4}{c|}{Abril}                                                                                & \multicolumn{4}{c|}{Mayo}                                                                                 & \multicolumn{4}{c|}{Junio}                                                                                                                                                     & \multicolumn{4}{c|}{Julio}                                                                                & \multicolumn{4}{c|}{Agosto}                                                                               \\ \cline{3-25} 
\multicolumn{1}{|c|}{\multirow{-2}{*}{Entregable}}                                  & \multirow{-2}{*}{\begin{tabular}[c]{@{}c@{}}Duración\\ Semanas\end{tabular}} & \multicolumn{1}{c|}{2}   & \multicolumn{1}{c|}{3}   & \multicolumn{1}{c|}{4}   & \multicolumn{1}{c|}{1}   & \multicolumn{1}{c|}{2}   & \multicolumn{1}{c|}{3}   & \multicolumn{1}{c|}{4}   & \multicolumn{1}{c|}{1}   & \multicolumn{1}{c|}{2}   & \multicolumn{1}{c|}{3}   & \multicolumn{1}{c|}{4}   & \multicolumn{1}{c|}{1}   & \multicolumn{1}{c|}{2}                          & \multicolumn{1}{c|}{3}                          & \multicolumn{1}{c|}{4}                          & \multicolumn{1}{c|}{1}   & \multicolumn{1}{c|}{2}   & \multicolumn{1}{c|}{3}   & \multicolumn{1}{c|}{4}   & \multicolumn{1}{c|}{1}   & \multicolumn{1}{c|}{2}   & \multicolumn{1}{c|}{3}   & \multicolumn{1}{c|}{4}   \\ \hline
Estudio Literatura                                                                  & 4                                                                            & \cellcolor[HTML]{FFFFC7} & \cellcolor[HTML]{FFFFC7} & \cellcolor[HTML]{FFFFC7} & \cellcolor[HTML]{FFFFC7} &                          &                          &                          &                          &                          &                          &                          &                          &                                                 &                                                 &                                                 &                          &                          &                          &                          &                          &                          &                          &                          \\ \hline
\begin{tabular}[c]{@{}l@{}}Instalar y desarrollar\\ software espectros\end{tabular} & 3                                                                            &                          &                          &                          & \cellcolor[HTML]{9AFF99} & \cellcolor[HTML]{9AFF99} & \cellcolor[HTML]{9AFF99} &                          &                          &                          &                          &                          &                          &                                                 &                                                 &                                                 &                          &                          &                          &                          &                          &                          &                          &                          \\ \hline
\begin{tabular}[c]{@{}l@{}}Adquisición y organización\\ de espectros\end{tabular}   & 4                                                                            &                          &                          &                          &                          &                          & \cellcolor[HTML]{9AFF99} & \cellcolor[HTML]{9AFF99} & \cellcolor[HTML]{9AFF99} & \cellcolor[HTML]{9AFF99} &                          &                          &                          &                                                 &                                                 &                                                 &                          &                          &                          &                          &                          &                          &                          &                          \\ \hline
Seleccionar mejores espectros                                                       & 3                                                                            &                          &                          &                          &                          &                          &                          &                          & \cellcolor[HTML]{9AFF99} & \cellcolor[HTML]{9AFF99} & \cellcolor[HTML]{9AFF99} &                          &                          &                                                 &                                                 &                                                 &                          &                          &                          &                          &                          &                          &                          &                          \\ \hline
Construir curvas de crecimiento                                                      & 3                                                                            &                          &                          &                          &                          &                          &                          &                          &                          &                          & \cellcolor[HTML]{DAE8FC} & \cellcolor[HTML]{DAE8FC} & \cellcolor[HTML]{DAE8FC} &                                                 &                                                 &                                                 &                          &                          &                          &                          &                          &                          &                          &                          \\ \hline
\begin{tabular}[c]{@{}l@{}}Comparación densidad columna\\ eclipse 1987\end{tabular} & 4                                                                            &                          &                          &                          &                          &                          &                          &                          &                          &                          &                          &                          &                          & \cellcolor[HTML]{DAE8FC}{\color[HTML]{DAE8FC} } & \cellcolor[HTML]{DAE8FC}{\color[HTML]{DAE8FC} } & \cellcolor[HTML]{DAE8FC}{\color[HTML]{DAE8FC} } & \cellcolor[HTML]{DAE8FC} &                          &                          &                          &                          &                          &                          &                          \\ \hline
Resultados física cromosférica                                                      & 4                                                                            &                          &                          &                          &                          &                          &                          &                          &                          &                          &                          &                          &                          &                                                 &                                                 &                                                 & \cellcolor[HTML]{DAE8FC} & \cellcolor[HTML]{DAE8FC} & \cellcolor[HTML]{DAE8FC} & \cellcolor[HTML]{DAE8FC} &                          &                          &                          &                          \\ \hline
Documentación del proyecto                                                          & 9                                                                            &                          &                          &                          &                          & \cellcolor[HTML]{FFFFC7} & \cellcolor[HTML]{FFFFC7} &                          &                          &                          &                          &                          & \cellcolor[HTML]{9AFF99} & \cellcolor[HTML]{9AFF99}                        &                                                 &                                                 &                          &                          &                          & \cellcolor[HTML]{DAE8FC} & \cellcolor[HTML]{DAE8FC} & \cellcolor[HTML]{DAE8FC} & \cellcolor[HTML]{DAE8FC} & \cellcolor[HTML]{DAE8FC} \\ \hline
\end{tabular}}
\caption{En amarillo, está la documentación, en verde la adquisición y organización de espectros y en azul los entregables más importantes}
\label{Cronograma}
\end{table}
%%%%%%%%%%%%%%%%%%%%%%%%%%%%%%%%%%%%%%%%%%%%%%%%%%%%%%%%%%%%%%%%%%%%%%%%%%%%%%%%%%%%%%%%%%%%%%

\section{Bibliografía}

\section{Anexos}
%%%%%%%%%%%%%%%%%%%%%%%%%%%%%%%%%%%%%%%%%%%%%%%%%%%%%%%%%%%%%%%%%%%%%%%%%%%%%%%%%%%%%%%%%%%%%%
\bibliographystyle{abbrv}
\bibliography{BiblioJen}

%%%%%%%%%%%%%%%%%%%%%%%%%%%%%%%%%%%%%%%%%%%%%%%%%%%%%%%%%%%%%%%%%%%%%%%%%%%%%%%%%%%%%%%%%%%%%%
\end{document}
